\documentclass[a4paper, 11pt]{article}
\pdfoutput=1
\usepackage{jheppub}

% uncomment this line for PRD format
% \documentclass[aps,prd,showpacs,letterpaper,onecolumn,longbibliography,superscriptaddress,notitlepage,nofootinbib]{revtex4-1}%

% already included in jheppub.sty. Uncomment for PRD format
%\usepackage{graphicx}
%\usepackage{amssymb}
%\usepackage{amsmath}
%\usepackage[colorlinks=true,citecolor=blue,linkcolor=blue, allcolors=blue]{hyperref}
%\usepackage{graphics,rotate,epsfig,color}

\usepackage[caption=false]{subfig}
\usepackage{amsfonts}
\usepackage{textcomp}
\usepackage{gensymb}
% defines table rules for professional looking tables
\usepackage{booktabs}

\usepackage[utf8]{inputenc}

\usepackage{listings}
\lstdefinestyle{mystyle}{
    basicstyle=\ttfamily,
    breakatwhitespace=false,
    breaklines=true,
    captionpos=b,
    keepspaces=true,
    %numbers=left,
    %numbersep=5pt,
    showspaces=false,
    showstringspaces=false,
    showtabs=false,
    tabsize=4
}
\lstset{style=mystyle}

%https://tex.stackexchange.com/questions/222697/how-to-use-booktab-or-type-this-table-in-revtex4-revtex4-1
\AtBeginDocument{
\heavyrulewidth=.08em
\lightrulewidth=.05em
\cmidrulewidth=.03em
\belowrulesep=.65ex
\belowbottomsep=0pt
\aboverulesep=.4ex
\abovetopsep=0pt
\cmidrulesep=\doublerulesep
\cmidrulekern=.5em
\defaultaddspace=.5em
}

\newcommand{\Python}{\texttt{Python}}
\newcommand{\like}{\mathcal{L}}
\newcommand{\vectheta}{\vec{\theta}}
\newcommand{\vecw}{\vec{w}}
\newcommand{\prob}{\mathcal{P}}
\newcommand{\gprob}{\mathcal{G}}
\newcommand{\meanl}{\mathcal{L}_{\textmd{Mean}}}
\newcommand{\mcl}{\like_\textmd{Eff}}
\newcommand{\gl}{\like_\textmd{G}}
\newcommand{\adhoc}{\mathcal{L}_{\textmd{AdHoc}}}
\newcommand{\lpoisson}{l_{\textmd{Poisson}}}
\newcommand{\lmc}{l_\textmd{Eff}}
\newcommand{\lbarlow}{\like_{\textmd{BB}}}
\newcommand{\hatmu}{\hat{\mu}}
\newcommand{\hatpoisson}{\hatmu_\textmd{Poisson}}
\newcommand{\hatmc}{\hatmu_\textmd{Eff}}
\newcommand{\au}{arb. unit}
\newcommand{\agpar}{\alpha}
\newcommand{\bgpar}{\beta}
\newcommand{\emcee}{\texttt{emcee}}
\newcommand{\meff}{m_\mathrm{Eff}}
\newcommand{\weff}{w_\mathrm{Eff}}

\DeclareMathOperator*{\argmax}{arg\,max}
\DeclareMathOperator*{\argmin}{arg\,min}

%%%%% TIKZ
\usepackage{tikz}
\usepackage{environ}

\usetikzlibrary{calc,trees,positioning,arrows,chains,shapes.geometric,%
    decorations.pathreplacing,decorations.pathmorphing,shapes,%
    matrix,shapes.symbols,backgrounds,fit} % required in the preamble
\usepackage{varwidth}

\makeatletter
\newsavebox{\measure@tikzpicture}
\NewEnviron{scaletikzpicturetowidth}[1]{%
  \def\tikz@width{#1}%
  \def\tikzscale{1}\begin{lrbox}{\measure@tikzpicture}%
  \BODY
  \end{lrbox}%
  \pgfmathparse{#1/\wd\measure@tikzpicture}%
  \edef\tikzscale{\pgfmathresult}%
  \BODY
}
\makeatother

\tikzset{
	%Define standard arrow tip
	>=stealth',
	%Define style for boxes
	box/.style={
		rectangle,
		rounded corners,
		dashed,
		draw=black, very thick,
		minimum height=2em,
		text centered,
		execute at begin node={\begin{varwidth}{28em}},
		execute at end node={\end{varwidth}}},
	solidbox/.style={
		rectangle,
		rounded corners,
		draw=black, very thick,
		minimum height=2em,
		text centered,
		execute at begin node={\begin{varwidth}{28em}},
			execute at end node={\end{varwidth}}},
    bigsolidbox/.style={
		rectangle,
		rounded corners,
		draw=black, very thick,
		minimum height=6cm,
		text centered,
		execute at begin node={\begin{varwidth}{28em}},
			execute at end node={\end{varwidth}}},
	% Define arrow style
	fw_arrow/.style={
		->,
		thick,
		shorten <=2pt,
		shorten >=2pt,},
	bw_arrow/.style={
		<-,
		thick,
		shorten <=2pt,
		shorten >=2pt,}
%	bigbox/.style={blue!50, thick, fill=blue!10, rounded corners, rectangle}
}

\definecolor{mc_gen_color}{RGB}{250,138,31}

\definecolor{det_sim_color}{RGB}{227,66,55}

\definecolor{llh_color}{RGB}{128,0,128}

\tikzstyle{bigboxGeneration} = [draw=mc_gen_color!50, thick, fill=mc_gen_color!20, rounded corners, rectangle]

\tikzstyle{bigboxDetector} = [draw=det_sim_color!50, thick, fill=det_sim_color!20, rounded corners, rectangle]

\tikzstyle{bigboxAnalysis} = [draw=llh_color!50, thick, fill=llh_color!20, rounded corners, rectangle]
   
\newcommand{\cmark}{\text{\ding{51}}}
\newcommand{\xmark}{\text{\ding{55}}}
%
