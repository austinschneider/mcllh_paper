The derivation above assumed identical weights. For arbitrary weights, $\mu$ is an outcome sampled from a compound Poisson distribution (CPD), which can be approximated by a scaled Poisson distribution (SPD) by matching the first and second moments of the two distributions~\cite{Bohm:2013gla}. In order to make the connection, first rewrite $\mu$ and $\sigma^2$ as
\begin{equation}\label{eq:effparameters}
\mu= \weff \meff~\textmd{and}~\sigma^2 = \weff^2 \meff,
\end{equation}
where $\meff$ is the effective number of MC events and $\weff$ the effective weight. From Eq.~\eqref{eq:ids} these are given by: $\meff = \mu^2/\sigma^2$ and $\weff=\sigma^2/\mu$. Next, assume $\bar m = \lambda/\weff$ and
\begin{align}
\label{eq:probmeff}
\like(\bar m|\meff)&= \frac{e^{-\bar m}{\bar m}^{\meff}}{\Gamma(\meff+1)},
\end{align}
where $\lambda$ again is the expected number of events in data. Equation \eqref{eq:probmeff} can be written as a likelihood function of $\lambda$,
\begin{equation}
\like(\lambda|\vecw(\vectheta))=\like(\lambda|\mu, \sigma)=\frac{e^{-\lambda\mu/\sigma^2}\left(\lambda\mu/\sigma^2\right)^{\mu^2/\sigma^2}}{\Gamma(\mu^2/\sigma^2+1)},
\label{eq:poisson_conditional_arb}
\end{equation}
which is identical to Eq.~\eqref{eq:poisson_conditional} except the denominator is now a gamma function instead of a factorial. However, since the denominator does not depend on $\lambda$ it cancels out in Eq.~\eqref{eq:posterior}.

To understand this approximation, note that the maximum likelihood in Eq.~\eqref{eq:probmeff} occurs when $\bar m = \meff$. The first and second moments of the SPD random variable $\weff M$, where $M \sim \mathrm{Poisson}(\meff)$, are given by
\begin{align}
\mathrm{E}[\weff M] &= \weff \meff \\
&= \mu, \nonumber
\end{align}
and
\begin{align}
\mathrm{Var}[\weff M] &= \weff^2 \meff \\
&= \sigma^2. \nonumber
\end{align}
This shows that the SPD, under the maximum likelihood solution for the given MC realization, has first and second moments that match the sample mean, $\mu$, and variance, $\sigma^2$, respectively. These are equal to the first and second moments of the CPD as described in~\cite{Bohm:2013gla}. By assuming that $\mu$ is drawn from a SPD, we can treat $\mu$ and $\sigma$ as outcomes that fix the likelihood function of the underlying scaled expectation $\lambda$, analogous to the case of identical weights. Because both the first and second moments are matched, this approximation accounts for the variance of the CPD unlike $\lbarlow$, which only accounts for the mean. Thus, while $\lbarlow$ is valid only for the case of narrow weight distributions, our approximation remains valid for broader distributions.
