The use of MC to estimate expected outcomes of physical processes is nowadays standard practice.
By construction, MC distributions are sample observations and subject to statistical fluctuations.
MC events are also typically weighted to a particular physics model, and these weights may not be uniform across all events in an observable bin.
A direct comparison of MC distributions to data is typically performed using $\adhoc$ or $\chi^2$, where the expectation from MC is computed as a sum over weights in a particular observable bin.
Such likelihoods neglect the intrinsic MC fluctuations and may lead to vastly underestimated parameter uncertainties in the case of low MC size.
A better approach is to use a likelihood that accounts for MC statistical uncertainties.

Along with the definitions of $\mu$ and $\sigma^2$ in Eq.~\eqref{eq:musigma}, the main result of this work is given in Eq.~\eqref{eq:parametrizedpoisson}.
This new $\mcl$ is motivated by treating the MC realization as an observation of a Poisson random variate, computing the likelihood of the expectation using the MC and marginalizing the Poisson probability of observed data over all possible expectations.
It is an analytic extension of the Poisson likelihood that accounts for MC statistical uncertainty under a uniform prior, $\prob(\lambda)$.
By assuming that the number of MC events per bin is the outcome of sampling a Poisson-distributed random variable, and that the SPD is a good approximation of the CPD for arbitrary weights, $\like \left(\lambda|\vecw(\vectheta)\right)$ can be written in terms of $\mu$ and $\sigma^2$ as shown in Eq.~\eqref{eq:poisson_conditional_arb}.
This allows us to calculate $\mcl$, given in Eq.~\eqref{eq:parametrizedpoisson}, which can be directly substituted in favor of $\adhoc$.
This construction is computationally efficient, exhibits proper limiting behavior, and has excellent coverage properties.
In the tests performed here, it outperforms other treatments of MC statistical uncertainty.
