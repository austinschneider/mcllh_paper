To treat MC statistical uncertainties in the small sample limit, a modification of the Poisson likelihood was introduced in~\cite{Barlow:1993dm}, which is briefly covered below.
First, note that the expectation in a single bin is given by contributions from different physical processes, which we index by $j$.
Then, the number of expected events can be written as
\begin{equation}
\lambda(\vectheta) = \sum_{j=1}^s \bar n_{j}(\vectheta),
\label{eq:barlow_true_nonweighted}
\end{equation}
where $\bar n_{j}$ is the expected number of MC events from process $j$ that fall in the bin and $s$ is the total number of relevant processes.
Substituting Eq.~\eqref{eq:barlow_true_nonweighted} into Eq.~\eqref{eq:poisson} gives the Poisson likelihood for observing $k$ data events.
For stochastic models, $\bar n_j$ is unknown.
Instead, the MC outcome can be modeled as having drawn $n_j$ events from a random process that simulates the physical process.
When MC generation is expensive, we can approximate $n_j$ as being drawn from a Poisson process with mean $\bar{n}_j$~\footnote{The MC generation is a binomial process where we generate a fixed number of events for each process, $N_j$, and accept them into the bin of interest with probability $\beta_{j}(\vectheta)$, such that $\bar n_{j}(\vectheta)=\beta_{j}(\vectheta) N_j$.
In the limit of both of rare processes ($\beta_{j} \ll 1$) and large number of generated events ($N_j \gg 1$), the total number of observed events can be approximated as Poisson distributed with mean $\lambda(\vectheta) = \sum_j\beta_{j}(\vectheta) N_j = \sum_j\bar n_{j}(\vectheta)$.}.
Profiling on the true number of MC events per process in the bin results in the Barlow-Beeston (BB) likelihood, given by~\cite{Barlow:1993dm}
\begin{equation}
\lbarlow(\vectheta|k) = \underset{\{\bar n_j\}}{\rm max}
\frac{\lambda(\vectheta)^{k}e^{-\lambda(\vectheta)}}{k!} \prod_{j=1}^s \frac{\bar n_j^{n_j}e^{-\bar n_j}}{n_j!},
\label{eq:barlow_likelihood_nonweighted}
\end{equation}
where $\lambda(\vectheta)$ is given by Eq.~\eqref{eq:barlow_true_nonweighted}, $n_{j}$ and $\bar{n}_{j}$ are the estimated and true MC counts in the bin respectively, and $\{\bar{n}_j\}_{j=1}^{s}$ denotes the $s$ nuisance parameters we have profiled over.

In the above formalism we have produced the MC at the natural rate, but this is not the case for weighted MC.
The prescription is given by replacing Eq.~\eqref{eq:barlow_true_nonweighted} with
\begin{equation}
\lambda(\vectheta) = \sum_{j=1}^s \eta_{j}(\vectheta) \bar n_j,
\label{eq:barlow_true_weighted}
\end{equation}
where $\eta_{j}(\vectheta)$ is a scale factor for process $j$ that accounts for the differences in the MC generation and the target hypothesis of interest.
In this case, the likelihood definition is still given by Eq.~\eqref{eq:barlow_likelihood_nonweighted}; an explicit formula for $s=1$ is given in the appendix.
However, for arbitrary weight distributions per physical process $\lbarlow$ may not be appropriate as it neglects the variance from a sum of weights~\cite{Barlow:1993dm}.
It remains valid only in the case where the distribution of weights for each process is narrow.

