\begin{figure}[ht]
\centering
    \includegraphics[width=1\linewidth]{fig/fig6}
\caption{\textbf{\textit{Posterior distributions in parameter space.}} Comparison of $\prob(\vectheta|k)$ for $\mcl$ (blue) and $\adhoc$ (orange).
Each horizontal row above uses a different MC set size, with $N_\mathrm{MC}=10^4$, $10^5$, and $10^6$ from top to bottom.
The left and center column show the marginal posterior distribution for the mass, $\Omega$, and normalization, $\Phi$, respectively.
The true value is indicated by the dashed, vertical line.
The rightmost column shows the joint posterior distribution with 68\% (solid) and 95\% (dashed) contours.
The true values are indicated by the star.}
\label{fig:llhdist}
\end{figure}

It is also possible to use $\mcl$ in a Bayesian approach.
Using Bayes' theorem, the posterior is
\begin{equation}
\prob(\vectheta|k) \propto \like(\vectheta|k) \pi(\vectheta),
\end{equation}
where $\pi(\vectheta)$ is a prior on the parameters.
As evaluation of the normalization factor can by challenging, $\prob(\vectheta|k)$ can be approximated using a Markov Chain Monte Carlo (MCMC).
For our toy example, we used \emcee{}~\cite{ForemanMackey:2012ig} to sample $\prob(\vectheta|k)$ under a uniform box prior for two different likelihood functions: $\mcl$ and $\adhoc$.
The sampling was performed using the data and MC sets described in Sec.~\ref{sec:pointestimation}.

Figure~\ref{fig:llhdist} shows the posterior distributions of $\Omega$ and $\Phi$.
For each comparison, $\mcl$ (blue) and $\adhoc$ (orange) were sampled using the same underlying data and MC.
We used 20 walkers with 300 burn-in steps followed by 1000 steps as settings for \emcee.
The left and center column show the marginal posterior distribution for the mass, $\Omega$, and normalization, $\Phi$, respectively.
The true value is indicated by the dashed, vertical line.
The rightmost column shows the joint posterior distribution with 68\% (solid) and 95\% (dashed) contours.
The true values are indicated by the star.
With $\adhoc$, the true value of the parameter is highly improbable for the lower MC-size cases of the top and middle rows.
In contrast, the posterior evaluated using $\mcl$ has increased width due to the reduced MC size.
Even for $N_\mathrm{MC}=10^6$ (bottom row), the shape of the posterior evaluated using $\adhoc$ is narrower than that using $\mcl$.
Credible regions estimated using $\adhoc$ would bias the result.

