\title{A binned likelihood for stochastic models}

\author[a,1]{C.A.~Arg\"uelles, \note{ORCID: \href{https://orcid.org/0000-0003-4186-4182}{0000-0003-4186-4182}}}
\author[b,2]{A.~Schneider, \note{ORCID: \href{https://orcid.org/0000-0002-0895-3477}{0000-0002-0895-3477}}}
\author[b,3]{T.~Yuan, \note{ORCID: \href{http://orcid.org/0000-0002-7041-5872}{0000-0002-7041-5872}}}
\affiliation[a]{Dept.~of Physics, Massachusetts Institute of Technology, Cambridge, MA 02139, USA}
\affiliation[b]{Dept.~of Physics and Wisconsin IceCube Particle Astrophysics Center, University of Wisconsin, Madison, WI 53706, USA}
\emailAdd{caad@mit.edu}
\emailAdd{aschneider@icecube.wisc.edu}
\emailAdd{tyuan@icecube.wisc.edu}

\keywords{Likelihood, Monte Carlo, Poisson distribution}
%\arxivnumber{abcd.xxxx}

\abstract{Metrics of model goodness-of-fit, model comparison, and model parameter estimation are the main categories of statistical problems in science. Bayesian and frequentist methods that address these questions often rely on a likelihood function, which is the key ingredient in order to assess the plausibility of model parameters given observed data. In some complex systems or experimental setups, predicting the outcome of a model cannot be done analytically, and Monte Carlo techniques are used. In this paper, we present a new analytic likelihood that takes into account Monte Carlo uncertainties, appropriate for use in the large and small sample size limits. Our formulation performs better than semi-analytic methods, prevents strong claims on biased statements, and provides improved coverage properties compared to available methods.}

